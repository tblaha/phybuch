\subsection{Gesetze im Kreis}

\subsubsection{Bahngeschwindigkeit im Kreis}

Die Bahngeschwindigkeit gibt in absoluten Einheiten ($\frac{m}{s}$) an, wie schnell sich das Objekt auf der Bahn fortbewegt. Zusätzlich zur Winkelgeschwindigkeit (\referenz{subsec:kenngroessen_schwingungen}) muss bei der Kreisbahn daher noch der Radius bekannt sein:

\begin{equation*} \label{eq:bahngeschwindigkeit}
	v=\frac{2r\pi}{T}=\omega r
\end{equation*}

Eine andere Herleitung aus dem Kreisumfang $U=2r\pi$, der Periodendauer $T$ und der generellen Formel für Bahngeschwindigkeit $v=\frac{s(t)}{t}$ könnte wie folgt Aussehen:

\begin{align*}
	v&=\frac{s(t)}{t} \\
	v&=\frac{2r\pi}{T}
\end{align*}


\subsubsection{Zentripetalbeschleunigung}

Um auf einer Kreisbahn zu bleiben, muss eine Zentripedalbeschleunigung auf einen Körper wirken, die zum Kreiszentrum hin zeigt. Die Formel lautet:

\begin{equation*}
	a_z=\frac{v^2}{r}=\omega^2 r
\end{equation*}


\subsubsection{Zentripetalkraft}

Dies ist die Kraft, die auf einen Körper ausgewirkt werden muss, damit er auf einer Kreisbahn bleibt. Zusätzlich zur Zentripetalbeschleunigung muss nun also noch gemäß Newtons zweiten Axiom $F=a \cdot m$ die Masse als Faktor in die Gleichung aufgenommen werden:

\begin{align*}
	F_z &= a_z \cdot m \\
	F_z &= \frac{v^{2}m}{r}
\end{align*}


\subsection{Gesetze bei Geschwindigkeiten}

Aus der maximalen Geschwindigkeit und der Frequenz, Periodendauer oder Winkelgeschwindigkeit lässt sich sofort auf die Amplitude schließen, da in der Funktion für die Geschwindigkeit bei einer Schwingung (\referenz{subsec:schwingungsgleichungen}) $v(t)=\omega \cdot y_{max} \cdot \cos{(\omega t)}$ das Maxima dann erreicht ist, wenn der Cosinus seinen Maximalwert $1$ annimmt. Dann gilt folgendes:

\begin{align*} \label{eq:geschwindigkeit_amplitude}
	v_{max} &= \omega \cdot y_{max} \\
	v_{max} &= 2\pi f \cdot y_{max} \\
	y_{max} &= \frac{v_{max}}{2\pi f}
\end{align*}


\subsection{Gesetze am Federpendel}

\subsubsection{Hooke'sches Gesetz}

Das Hooke'sche Gesetz gibt die Federhärte einer Feder, also dessen Kenngröße, an. Die Einheit $\frac{N}{m}$ erklärt selbst die Formel:

\begin{equation*}
	D=\frac{F}{l}
\end{equation*}

Mit dem Gesetz ist gezeigt, dass an einem Federpendel die Rückstellkraft $F$ proportional zur Auslenkung $l$ ist. Damit ist es eine harmonische Schwingung. (\referenz{subsec:definitionenzuschwingungen})

\subsubsection{Periodendauer beim Federpendel}

Die Periodendauer beim Federpendel ist abhängig von der Masse $m$ und der Federkonstante $D$:

\begin{equation*}
	T_{Feder}=2\pi \cdot \frac{m}{D}
\end{equation*}

Die Herleitung gestaltet sich folgendermaßen: Die zur Auslenkung $y$ proportionale Kraft $F$ ist die Rückstellende Kraft; beschrieben durch eine Umformung des Hooke'schen Gesetzes: $F_{r}=D \cdot y$. 
%Der Vektor dieser Rückstellende Kraft zeigt dem der Auslenkenden Kraft genau entgegen, daher muss der Term in diesem Fall negiert werden: $F_{r}=-D \cdot y$.

In einem geschlossenen System ist die Summe aller Kräfte $0$. Es ergibt sich Folgendes:

\begin{align*}
	F_a + F_r &= 0 \\
	F_a &= -F_r \\
\end{align*}

Die Kraft $F_a$ kann wie jede Kraft mit Newtons zweitem Gesetz als $F_a=m \cdot a$ beschrieben werden. Die Beschleunigung in $a$ kann dann als zweite Ableitung der Auslenkung $y$ nach der Zeit beschrieben werden. Aus den Schwingungsgleichungen (\referenz{subsec:schwingungsgleichungen}) geht hervor, dass $y''=-\omega^{2} \cdot y$ ist. Daher kann man wie folgt einsetzen:

\begin{align*}
	m \cdot a &= -D \cdot y \\
	m \cdot y'' &= -D \cdot y \\
	m \cdot -\omega^{2} \cdot y &= -D \cdot y \\
	- m \cdot \omega^{2} &= -D \\
	\omega &= \sqrt{\frac{D}{m}}
\end{align*}

Für aus $\omega=\frac{2\pi}{T}$ folgt für $T$:

\begin{equation*}
	T = 2\pi \cdot \sqrt{\frac{m}{D}}
\end{equation*}
