\subsection{Gesetze bei Geschwindigkeiten}

Aus der maximalen Geschwindigkeit und der Frequenz, Periodendauer oder Winkelgeschwindigkeit lässt sich sofort auf die Amplitude schließen, da in der Funktion für die Geschwindigkeit bei einer Schwingung (\referenz{sec:schwingungsgleichungen}) $v(t)=\omega \cdot y_{max} \cdot \cos{(\omega t)}$ das Maxima dann erreicht ist, wenn der Cosinus seinen Maximalwert $1$ annimmt. Dann gilt folgendes:

\begin{align} \label{eq:geschwindigkeit_amplitude}
\begin{split}
	v_{max} &= \omega \cdot y_{max} \\
	v_{max} &= 2\pi f \cdot y_{max} \\
	y_{max} &= \frac{v_{max}}{2\pi f}
\end{split}
\end{align}


\subsection{Gesetze am Federpendel} \label{subsec:gesetze_federpendel}

\subsubsection{Hooke'sches Gesetz}

Das Hooke'sche Gesetz gibt die Federhärte einer Feder, also dessen Kenngröße, an. Die Einheit $\frac{N}{m}$ erklärt selbst die Formel:

\begin{equation}
	D=\frac{F}{l}
\end{equation}

Mit dem Gesetz ist gezeigt, dass an einem Federpendel die Rückstellkraft $F$ proportional zur Auslenkung $l$ ist, da $D$ konstant ist. Damit handelt es sich um eine harmonische Schwingung (\referenz{subsec:harmonisch}).

\subsubsection{Periodendauer beim Federpendel}

Die Periodendauer beim Federpendel ist abhängig von der Masse $m$ und der Federkonstante $D$:

\begin{equation}
	T_{Feder}=2\pi \cdot \frac{m}{D}
\end{equation}

Die Herleitung gestaltet sich folgendermaßen: Die zur Auslenkung $y$ proportionale Kraft $F$ ist die Rückstellende Kraft; beschrieben durch eine Umformung des Hooke'schen Gesetzes: $F_{r}=D \cdot y$. 

In einem geschlossenen System ist die Summe aller Kräfte $0$. Es ergibt sich Folgendes:

\begin{align}
\begin{split}
	F_a + F_r &= 0 \\
	F_a &= -F_r \\
\end{split}
\end{align}

Die auslenkende Kraft $F_a$ kann wie jede Kraft mit Newtons zweitem Gesetz als $F_a=m \cdot a$ beschrieben werden. Die Beschleunigung in $a$ kann dann als zweite Ableitung der Auslenkung $y$ nach der Zeit beschrieben werden. Aus den Schwingungsgleichungen (\referenz{sec:schwingungsgleichungen}) geht hervor, dass $y''=-\omega^{2} \cdot y$ ist. Daher kann man wie folgt einsetzen:

\begin{align}
\begin{split}
	m \cdot a &= -D \cdot y \\
	m \cdot y'' &= -D \cdot y \\
	m \cdot -\omega^{2} \cdot y &= -D \cdot y \\
	- m \cdot \omega^{2} &= -D \\
	\omega &= \sqrt{\frac{D}{m}}
\end{split}
\end{align}

Für aus $\omega=\frac{2\pi}{T}$ folgt für $T$:

\begin{equation}
	T = 2\pi \cdot \sqrt{\frac{m}{D}}
\end{equation}
