Daraus ergibt sich folgende Schwingungsgleichung, bzw. Schwingungsfunktion, für eine harmonische Schwingung (\referenz{subsec:harmonisch}), die die Elongation in Abhängigkeit der Zeit angibt:

\begin{equation} \label{eq:schwingungsgleichung_y}
	y(t)=y_{max} \cdot \sin{(\omega t + \phi)}
\end{equation}

Die erste Ableitung dieser Gleichung nach $t$ gibt die Geschwindigkeit des Schwing-körpers zur Zeit $t$ an:

\begin{equation} \label{eq:schwingungsgleichung_v}
	y'(t)=v(t)=y_{max} \cdot \omega \cdot \cos{(\omega t + \phi)}
\end{equation}

Die zweite Ableitung gibt die Beschleunigung des Schwingkörpers zur Zeit $t$ an:

\begin{equation} \label{eq:schwingungsgleichung_a}
	y''(t)=v'(t)=a(t)=y_{max} \cdot \omega^{2} \cdot -\sin{(\omega t + \phi)}
\end{equation}