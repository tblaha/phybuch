Als Schwingung bezeichnet man eine periodische Energieumwandlung von einer Energieform in eine andere und umgekehrt. Im einfachsten Fall ist eine Feder zu betrachten, an der ein Massestück hängt. Wird diese beispielsweise nach unten ausgelenkt (an ihr gezogen und dann losgelassen), schwingt sie periodisch und wandelt dabei immerfort kinetische Energie in potentielle Energie um, beziehungsweise umgekehrt.


\subsection{Auslenkung}

Den Abstand des Schwingkörpers von seiner Ruhelage nennt man Auslenkung oder Elongation.

Oft wird sie in Formel mit $s$ geschrieben.


\subsection{Amplitude}

Die Amplitude ist die maximale Auslenkung, die während der Schwingung in jeder der beiden Schwingungsrichtungen auftritt.

\subsection{Rückstellkraft}

Die Kraft, die immer der Auslenkungsrichtung entgegen wirkt, also den Körper in Richtung der Ruheposition beschleunigt und so überhaupt erst eine Schwingung ermöglicht, nennt man Rückstellkraft.

In Formeln wird oft $F_r$ verwendet.


\subsection{Harmonische Schwingung} \label{subsec:harmonisch}

Der zeitliche Verlauf einer harmonischen Schwingung kann mit einer Sinuswelle beschrieben werden. Das mathematische Kriterium ist die Proportionalität der Rückstellkraft $F_{r}$ zum Betrag der Auslenkung $l$:

\begin{equation} \label{eq:kriterium_harmonisch}
	F_{r} \sim l
\end{equation}


\subsection{Gedämpfte Schwingung}

Bei einer gedämpften Schwingung nimmt die Amplitude über der Zeit ab. Außerhalb des im Physikunterricht angenommenen Modells sind alle Schwingung gedämpft, da Energie auch an die Umgebung abgegeben wird, beispielsweise, durch Reibung, in Form von thermischer Energie.

