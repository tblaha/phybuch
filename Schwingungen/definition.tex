Als Schwingung bezeichnet man eine periodische Energieumwandlung von einer Energieform in eine andere und umgekehrt immerfort.

\subsection{Harmonische Schwingung}

Der zeitliche Verlauf einer harmonische Schwingung kann mit einer Sinuswelle beschrieben werden. Das mathematische Kriterium ist die Proportionalität der Rückstellkraft $F_{r}$ zum Betrag der Auslenkung $l$:

\begin{equation} \label{eq:kriterium_harmonisch}
	F_{r} \sim l
\end{equation}


\subsection{Gedämpfte Schwingung}

Bei einer gedämpften Schwingung nimmt die Elongation über der Zeit ab. Außerhalb des im Physikunterricht angenommenen Modells sind alle Schwingung gedämpft, da Energie auch an die Umgebung abgegeben wird, beispielsweise durch Reibung in thermische Energie.

