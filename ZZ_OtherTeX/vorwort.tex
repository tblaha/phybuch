\section*{Lizenzen}

\subsubsection*{Dokument}

Der Autor übernimmt keine Garantie, in welchem Zusammenhang auch immer, für die Richtigkeit, Vollständigkeit oder Verständlichkeit dieses Dokuments. Es ist nur als Zusammenfassung und Unterstützung beim Lernen gedacht, nicht als alleiniges Lehrmaterial.

Dieses Dokument ist unter der \textsc{GNU Free Documentation License}\footnote{Englischer Orginaltext zu \textsc{GFDL} liegt bei oder ist hier erhältlich: \url{https://www.gnu.org/licenses/fdl.txt}} lizenziert. Diese stellt es jedem frei, dieses Dokument unter gewissen Auflagen zu vertreiben oder zu ändern. Diese Auflagen beinhalten zum Beispiel, dass das weitervertriebene Produkt auch frei sein muss und dass bei einer Änderung des Dokumentes der Name des Autors und ein Beschreibung der Änderung beigefügt werden muss. Für genauere Details ziehen Sie bitte den originalen Lizenztext heran.

Für die Bereitstellung des \LaTeX{} Quellcodes zum Ändern oä. bitte ich, mich per Email direkt zu kontaktieren.

Der Autor ist außerdem für jegliche Meinungen, Korrekturen, Anregungen und wüste Beschimpfungen offen:

\href{mailto:till.blaha@web.de}{till.blaha@web.de}


\subsubsection*{Illustrationen und Zitate}

Die Autoren und Lizenzen der in diesem Dokument benutzten Illustration und Zitate wurden mit Endnoten angegeben. Diese finden sich ab Seite \pageref{endnotes}.

Der Orginaltext der oft verwendeten \glqq Creative-Commons\grqq -Lizenz \textsc{CC BY-SA 3.0} liegt ebenfalls bei oder ist Online erhältlich\footnote{Englischer Orginaltext zu \textsc{CC BY-SA 3.0}: \url{https://creativecommons.org/licenses/by-sa/3.0/legalcode}}


\section*{Umfang}

Dieses Material basiert auf dem Umfang des \glqq Regionalcurriculum für das Fach Physik, Prüfungsregion 12\grqq{}\footnote{Orginaltext: \url{http://www.idsb.eu/intranet/teamworks.dll/Internal/document/0xD4A9A3187641B24EB79D9320B3060289/SC-RC\%20Physik\%2010-12.pdf}} mit Ausnahme der Rubrik \glqq Herz'sche Wellen\grqq , \glqq Quantenphysik der Elektronenhülle\grqq{} und \glqq Physik des Atomkerns\grqq .


\section*{Dank}

Besonderer Dank geht an Leonard Geilen, der viel zu den Kapiteln Optik und Quanten beigetragen hat! Außerdem möchte ich Herrn Kunz herzlich für den hervorragenden Physikunterricht danken, der mich zum Schreiben dieser Zusammenfassung inspiriert hat. 


\section*{Version}

Bei diesem Dokument handelt es sich um die Version \version , kompiliert am \today .