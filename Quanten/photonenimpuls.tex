Der Impuls $\vec{p}$ eines beliebigen Körpers mit Masse ist definiert als $\vec{p} = m \cdot \vec{v}$ (Siehe: \referenz{subsec:impuls}). 

Das Problem bei der Anwendung auf Photonen ist, dass Photonen keine Ruhemasse haben, ihre effektive Masse lässt sich nur durch Umformen der Gleichung der Einstein'schen Relativitätstheorie ermitteln:

\begin{align}
\begin{split}
	E &= mc^2 \\
	m &= \frac{E}{c^2}
\end{split}
\end{align}

\noindent Hierbei ist $c$ die Lichtgeschwindigkeit mit der sich Photonen bewegen. Daher folgt:

\begin{align}
\begin{split}
	p &= m \cdot v \\
	p &= \frac{E}{c^2} \cdot c \\
	p &= \frac{E}{c}
\end{split}
\end{align}

\noindent Mit der Photonenenergie (Siehe: \gleichungsreferenz{eq:einsteinscheGleichung}) und der Beziehung für die Ausbreitungsgeschwindigkeit $c=\lambda \cdot f$ gilt:

\begin{align}	\label{eq:Photonenimpuls}
\begin{split}
	p &= \frac{h \cdot f}{c} = \frac{h}{\lambda}
\end{split}
\end{align}

\begin{NiceToKnow}
	Für viele Berechnungen mit dem Photonenimpuls: $F = \frac{\Delta p}{\Delta t}$
\end{NiceToKnow}


\subsection{Elastischer Stoß}

Bei einem elastischen Stoß bleibt, neben der Summe des Impulses, die Summe der kinetischen Energien der Stoßpartner unverändert (Siehe: \referenz{sec:stoesse}).

Für Photonen kann man dann folgern: wenn $m_1 << m_2$ (wie beim Vorgang: Photon trifft auf Reflektor) ist nach dem Stoß der Betrag der Geschwindigkeit des Photon in guter Näherung dieselbe wie zu Beginn, aber in die entgegengesetzte Richtung: $v'_1 \approx -v_1$. Der Impulsübertrag ist also $2p$.


\subsection{Unelastischer Stoß}

Dieser Vorgang ist nicht reibungsfrei, es geht kinetische Energie "verloren" (Umwandlung in Wärme).

Ein Beispiel wäre, wenn Photonen auf eine schwarze Fläche treffen: es kommt nur zum einfachen Impulsübertrag, die restliche Energie wird in Wärme umgewandelt.
