\begin{itemize}
\item Der Impuls $\vec{p}$ eines beliebigen Körpers mit Masse ist definiert als: 

$\vec{p} = m \cdot \vec{v}$
\item Photonen haben aber keine Ruhemasse. Ihre Masse lässt sich nur durch Umformen der Gleichung der speziellen Relativitätstheorie $E = mc^2$ ermitteln.

$m = \frac{E}{c^2}$

daraus folgt:

$\vec{p} = \frac{E}{c^2} \cdot \vec{c}$ \tabto{0.5\textwidth} ; $\vec{v} = \vec{c}$ da Photonen sich stets mit Lichtgeschwindigkeit fortbewegen.

$\vec{p} = \frac{h \cdot f}{\vec{c}}$ \tabto{0.5\textwidth} ; $E_{phot} = h \cdot f$

$\vec{p} = \frac{h}{\lambda \cdot f}$ \tabto{0.5\textwidth} ; $c = \lambda \cdot f$

$\vec{p} = \frac{h}{\lambda}$
\item Das heißt das, dass alles mit einem Impuls eine Wellenlänge hat, \referenz{subsec:DeBroglie}
\item Nice to know: $F = \frac{\Delta p}{\Delta t}$
\end{itemize}

\subsection{Elastischer Stoß}
\begin{itemize}
\item Bei einem elastischen Stoß bleibt die Summe der kinetischen Energien der Stoßpartner unverändert.

$\frac{1}{2}m_1 \cdot v_1^2 = \frac{1}{2}m_1 \cdot (v'_1)^2 + \frac{1}{2}m_2 \cdot (v'_2)^2$
\item Wenn $m_1 << m_2$ (wie bei Experiment Lichtmühle bzw. Photon Reflektor) ist $v'_1 \approx -v_1$ und $\vec{p'} \approx -2 \vec{p}$
\end{itemize}

\subsection{Unelastischer Stoß}
\begin{itemize}
\item Der Vorgang ist nicht reibungsfrei, es geht kinetische Energie "verloren" (Umwandlung in Wärme).
\item Beispiel Elektronen auf schwarze Fläche, es kommt nur zum einfachen Impulsübertrag, die verlorene Energie wird in Wärme umgewandelt.
\end{itemize}


