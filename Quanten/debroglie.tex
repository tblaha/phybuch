Der Physiker Louis De Broglie erkannte aus der Impulsgleichung des Photons (\gleichungsreferenz{eq:Photonenimpuls}), dass mathematische gesehen, jedes Teilchen mit einem Impuls, also jedes Teilchen, bzw. jeder Körper, mit einer Masse und einer Geschwindigkeit, eine Wellenlänge hat.

Diese Art der Welle nennt man auch Materialwelle oder De-Broglie-Welle.

Die Berechnung der De-Broglie-Wellenlänge gestaltet sich denkbar einfach. Aus Gleichung \ref{eq:Photonenimpuls} geht hervor:

\begin{equation}
	\lambda_{dB} = \frac{h}{p} = \frac{h}{m \cdot v}
\end{equation}