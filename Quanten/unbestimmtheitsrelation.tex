Werner Heisenberg fand heraus, dass man bei Quantenobjekten, z.B. Elektronen, den Aufenthaltsort \emph{und} den Impuls, also die Geschwindigkeit und die Richtung (Im Impulse steckt die Geschwindigkeit, eine gerichtete Größe) nicht genau bestimmen kann. Das hat nichts mit den Ungenauigkeiten von Messtechniken zu tun, sondern ist ein Phänomen.

Er stellte die Heisenbergsche Unbestimmtheitsrelation auf, auch Unschärferelation genannt:

\begin{align}
\begin{split}
	\Delta x \cdot \Delta p \ge \frac{h}{4\pi}
\end{split}
\end{align}

Hierbei ist $\Delta x$ die Ortsunschärfe und $\Delta p$ die Impulsunschärfe. Die sind die Genauigkeitsbereiche, mit denen die beiden Größen bestimmt werden können. Wenn man also nun den Ort genau bestimmen kann, kann man den Impuls bestimmen und daher nicht genau vorhersagen, wie schnell oder in welche Richtung sich das Teilchen fortbewegen wird.

Auf Elektronen am Doppelspalt angewendet bedeutet das, dass man, auch wenn der Ort des Elektrons sehr genau bestimmt werden kann (es kann sich ja nur durch einen der beiden Spalte bewegen und dort kann man Detektoren anbringen), nicht sicher weiß, wohin es sich bewegt, da man zwar weiß, aus welcher Richtung es kam, aber die Ablenkung am Doppelspalt und daher den Impuls und die weitere Flugbahn, nicht genau bestimmen kann.