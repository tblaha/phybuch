Mit dem Hallwachs Experiment entdeckte Wilhelm Hallwachs 1888 den fotoelektischen Effekt, der die klassische (mechanische) Wellentheorie widerlegte und das Kapitel der Quantenphysik öffnete.

\subsection{Aufbau und Beobachtung}

Die geladene Zinkplatte eines Ladungsmesser (Siehe: \referenz{subsec:Elektroskop}) wird zunächst mit weißem Licht bestrahlt, wobei sich keine Veränderung des Ausschlags der Nadel zeigt. Auch beim Hinzunehmen weiter, identischer Lichtquellen, also einer Erhöhung der Intensität des Lichts, gibt es keine Veränderung.

Beim Wiederholung des Versuches mit Ultraviolettlicht, welches ein kürzere Wellenlänge besitzt, wird allerdings der Ausschlag der Nadel kleiner.

\subsection{Deutung nach Hallwachs}

Elektronen wurden aus der Metallplatte gelöst, daher kommt die Veränderung des Nadelausschlags. Daraus folgt, dass die Energie von elektromagnetischen Wellen abhängig von der Wellenlänge ist und \textbf{nicht} von der Intensität. \textbf{Dies ist ein Widerspruch zur klassischen Wellentheorie.}

Licht muss also neben dem Wellencharakter (Siehe: \kapitelreferenz{ch:Optik}) auch Teilchencharakter zeigen. Dieses Lichtteilchen nennt man \glqq Photon\grqq .