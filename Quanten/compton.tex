Wenn man Photonen mit ausreichend Energie (ab ca. 100keV) auf gebundene Elektronen schießt, wird das Phänomen durch den Fotoeffekt (Siehe \referenz{sec:fotoeffekt}) beschrieben. Es kommt zum vollständigen Energieübertrag vom Photon auf das Elektron.

Wenn man ein Photon nun gegen ein freies Elektron schießt, wird das Phänomen durch den Compton-Effekt beschrieben:

\subsection{Effekt}

Es kommt nicht immer zu einer vollständigen Impulsübertrag, nämlich kann das Elektron auch zur Seite abgelenkt und das Photon dann in der Konsequenz nicht in die entgegengesetzte Richtung reflektiert. Wenn nicht der ganze Impuls übertragen wird, kommt es zur Compton-Streuung. 

Das Elektron wird durch den Teilimpuls in eine Richtung beschleunigt, während das Photon in eine andere abgelenkt wird. Nach dem Zusammenstoß hat das Photon einen geringeren Impuls und da seine Geschwindigkeit konstant $c$ ist, ändert sich, bedingt durch den Impulserhaltungssatz, seine Wellenlänge (sie wird größer).

\begin{comment}\item Durch den Impulserhaltungssatz gilt:\\
Vor Kollision:\\
$p_{gesamt} = p_{Phot} \rightarrow p_{gesamt}=\frac{h}{\lambda}$


Nach Kollision: \\
$p_{gesamt} = p'_{phot} + p_e$

$\frac{h}{\lambda} = \frac{h}{\lambda'} + m_e \cdot v$

$\frac{h}{\lambda} = \frac{h}{\lambda} \cdot \cos{\Theta} + \frac{h}{\lambda} \cdot (1-\cos{\Theta})$ \tabto{0.5\textwidth} ; $c = \lambda * f \rightarrow \lambda = c / f$

$\frac{h}{\lambda} = \frac{h}{\lambda} \cdot \cos{\Theta} + \frac{h}{\lambda} \cdot (1-\cos{\Theta})$





$m_e \cdot v = \frac{h}{\lambda}$ \tabto{0.5\textwidth} ; umstellen nach $\lambda$ 


$\lambda' = \lambda + \Delta\lambda$

$\lambda' - \lambda = \Delta\lambda$

$\lambda' - \lambda = $


$\lambda = \frac{h}{m_e \cdot v}$

\end{comment}

\begin{comment}
Vor Kollision:\\
$p_{gesamt} = p_{phot}$

Nach Kollision: \\
$p_{gesamt} = p'_{phot} + p_{Teilchen, an dem gestreut}$

Für $\Delta p$: \\
$p_{gesamt} - p'_{phot} = \Delta p$ \tabto{0.5\textwidth} ; umstellen nach $p_{gesamt}$ \\
$p_{gesamt} = \Delta p + p'_{phot}$ \\

Einsetzen in die Gesamtgleichung für nach Kollision:\\
$\frac{h}{\Delta \lambda} + \frac{h}{\lambda '} = \frac{h}{\lambda '} + m \cdot v$ \tabto{0.5\textwidth} ; mit $p_{Teilchen, an dem gestreut} = m \cdot v$ \\
$\frac{h}{\Delta \lambda} = m \cdot v$ \\
$\frac{h}{\Delta \lambda \cdot m \cdot v} = 1$ \\
$\Delta \lambda = \lambda_c = \frac{h}{m \cdot v}$ \\


\end{comment}

\begin{comment}
\item Compton-Wellenlänge hergeleitet:

Durch den Impulserhaltungssatz gilt:\\

Vor Kollision:\\
$p_{gesamt} = p_{phot}$

Nach Kollision: \\
$p_{gesamt} = p'_{phot} + p_{Teilchen, \ an \ dem \ gestreut}$

Für $\Delta p$: \\
$\Delta p = p_{phot} - p'_{phot}$

$\Delta p = p_{gesamt} - p'_{phot}$ \tabto{0.5\textwidth} ; umstellen nach $p_{gesamt}$

$p_{gesamt} = \Delta p + p'_{phot}$ \\


Einsetzen in die Gesamtgleichung für "nach Kollision": \\
$\frac{h}{\Delta \lambda} + \frac{h}{\lambda '} = \frac{h}{\lambda '} + m \cdot v$ \tabto{0.5\textwidth} ; mit $p_{Teilchen, \ an \ dem \ gestreut} = m \cdot v$

$\frac{h}{\Delta \lambda} = m \cdot v$ \\

Dann noch nach $\Delta \lambda$ umformen und man erhält die Compton Wellenlänge: \\
$\Delta \lambda = \lambda_c = \frac{h}{m \cdot v}$ \\
\end{comment}

Für ein Elektron als Teilchen, an dem gestreut wird, gilt die Compton-Wellenlänge des Elektrons (\casio{12}):

\begin{align}
	\Delta \lambda = \lambda_c = \frac{h}{m_e \cdot v} \approx 2,426 \cdot 10^{-12}m
\end{align}

Wenn $\lambda$ die Wellenlänge des Lichts vor dem Stoß und $\lambda'$ die Wellenlänge nach dem Stoß, dann ist die Wellenlängenänderung, in Abhängigkeit des Streuungswinkels $\Theta$:

\begin{align}
\begin{split}
	\lambda' - \lambda &= \frac{h}{m_e \cdot c}-\frac{h}{m \cdot c} \cos{\Theta} \\
	\Delta\lambda &= \frac{h}{m_e \cdot c}(1-\cos{\Theta}) \\
	\Delta\lambda &= \lambda_c \cdot (1-\cos{\Theta})
\end{split}
\end{align}


\noindent Das Maximum dieser Gleichung liegt bei $\Theta = 180\degree \rightarrow (1-\cos{180\degree})=2$. \\
Das Minimum ist bei $\Theta = 0\degree \rightarrow (1-\cos{0\degree})=0$.
