\begin{itemize}
\item Wenn man Photonen mit ausreichend Energie (ab ca. 100keV) auf gebundene Elektronen schießt, wird das Phänomen durch den Photoeffekt beschrieben. Es kommt zur vollständigen Impulsübertragung vom Photon auf das Elektron.
\item Wenn man Photonen gegen ein freies Elektron schießt, wird das Phänomen durch den Compton-Effekt beschrieben:
\begin{itemize}
\item Es kommt nicht immer zu einer vollständigen Impulsübertrag.
\item Wenn nicht der ganze Impuls übertragen wird, kommt es zur Compton-Streuung.
Das Elektron wird in eine Richtung wird durch den Teilimpuls in eine Richtung beschleunigt, und das Photon wird in eine andere abgelenkt.\\
Nach dem Zusammenstoß hat das Photon einen geringeren Impuls, und da seine Geschwindigkeit konstant $c$ ist, ändert sich seine Wellenlänge (sie wird größer).
\begin{comment}\item Durch den Impulserhaltungssatz gilt:\\
Vor Kollision:\\
$p_{gesamt} = p_{Phot} \rightarrow p_{gesamt}=\frac{h}{\lambda}$


Nach Kollision: \\
$p_{gesamt} = p'_{phot} + p_e$

$\frac{h}{\lambda} = \frac{h}{\lambda'} + m_e \cdot v$

$\frac{h}{\lambda} = \frac{h}{\lambda} \cdot \cos{\Theta} + \frac{h}{\lambda} \cdot (1-\cos{\Theta})$ \tabto{0.5\textwidth} ; $c = \lambda * f \rightarrow \lambda = c / f$

$\frac{h}{\lambda} = \frac{h}{\lambda} \cdot \cos{\Theta} + \frac{h}{\lambda} \cdot (1-\cos{\Theta})$





$m_e \cdot v = \frac{h}{\lambda}$ \tabto{0.5\textwidth} ; umstellen nach $\lambda$ 


$\lambda' = \lambda + \Delta\lambda$

$\lambda' - \lambda = \Delta\lambda$

$\lambda' - \lambda = $


$\lambda = \frac{h}{m_e \cdot v}$

\end{comment}

\begin{comment}
Vor Kollision:\\
$p_{gesamt} = p_{phot}$

Nach Kollision: \\
$p_{gesamt} = p'_{phot} + p_{Teilchen, an dem gestreut}$

Für $\Delta p$: \\
$p_{gesamt} - p'_{phot} = \Delta p$ \tabto{0.5\textwidth} ; umstellen nach $p_{gesamt}$ \\
$p_{gesamt} = \Delta p + p'_{phot}$ \\

Einsetzen in die Gesamtgleichung für nach Kollision:\\
$\frac{h}{\Delta \lambda} + \frac{h}{\lambda '} = \frac{h}{\lambda '} + m \cdot v$ \tabto{0.5\textwidth} ; mit $p_{Teilchen, an dem gestreut} = m \cdot v$ \\
$\frac{h}{\Delta \lambda} = m \cdot v$ \\
$\frac{h}{\Delta \lambda \cdot m \cdot v} = 1$ \\
$\Delta \lambda = \lambda_c = \frac{h}{m \cdot v}$ \\


\end{comment}
\end{itemize}
\begin{comment}
\item Compton-Wellenlänge hergeleitet:

Durch den Impulserhaltungssatz gilt:\\

Vor Kollision:\\
$p_{gesamt} = p_{phot}$

Nach Kollision: \\
$p_{gesamt} = p'_{phot} + p_{Teilchen, \ an \ dem \ gestreut}$

Für $\Delta p$: \\
$\Delta p = p_{phot} - p'_{phot}$

$\Delta p = p_{gesamt} - p'_{phot}$ \tabto{0.5\textwidth} ; umstellen nach $p_{gesamt}$

$p_{gesamt} = \Delta p + p'_{phot}$ \\


Einsetzen in die Gesamtgleichung für "nach Kollision": \\
$\frac{h}{\Delta \lambda} + \frac{h}{\lambda '} = \frac{h}{\lambda '} + m \cdot v$ \tabto{0.5\textwidth} ; mit $p_{Teilchen, \ an \ dem \ gestreut} = m \cdot v$

$\frac{h}{\Delta \lambda} = m \cdot v$ \\

Dann noch nach $\Delta \lambda$ umformen und man erhält die Compton Wellenlänge: \\
$\Delta \lambda = \lambda_c = \frac{h}{m \cdot v}$ \\
\end{comment}

Für ein Elektron als Teilchen, an dem gestreut wird, gilt folgende Konstante (Taschenrechner: Konstante 12: "Compton-Wellenlänge des Elektrons"): \\
$\Delta \lambda = \lambda_c = \frac{h}{m_e \cdot v} \approx 2,426 \cdot 10^{-12}m $ \\


\item Wellenlängenänderung in Abhängigkeit des Streuungswinkels $\Theta$:

$\lambda' -\lambda = \frac{h}{m \cdot c}-\frac{h}{m \cdot c} \cos{\Theta}$

$\Delta\lambda = \frac{h}{m \cdot c}(1-\cos{\Theta})$ \tabto{0.5\textwidth} ; $\lambda_c = \frac{h}{m \cdot c}$

$\Delta\lambda = \lambda_c \cdot (1-\cos{\Theta})$

\item Maximum ist bei $\Theta = 180\degree \rightarrow (1-\cos{180\degree})=2$
\item Minimum ist bei $\Theta = 0\degree \rightarrow (1-\cos{0\degree})=0$

\end{itemize}