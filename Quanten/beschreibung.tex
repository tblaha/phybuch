Als man den Versuch mit Elektronen am Doppelspalt durchführte, tat man dies auch mit einzelnen Elektronen, die nacheinander auf den Spalt \glqq abgefeuert\grqq{} wurden und stellte fest, dass auch bei dieser Konfiguration das Interferenzmuster auftrat.

Um den Versuch noch genauer zu untersuchen und zu überwachen, brachte man einen Detektor an einem der Spalte an, um festzustellen, durch welchen Spalt sich das Elektron bewegt. Als man dies aber tat, traten keine Interferenzen mehr auf, was man am Bild der Elektronen auf dem Schirm erkennen konnte.

Diese Entdeckung ließ und lässt sehr viele Fragen auf, aber stellte klar, dass man das Verhalten von Quantenobjekte nicht vorhersagen und nicht einmal vollständig beschreiben kann, sondern Aussagen nur mit Wahrscheinlichkeiten treffen kann.