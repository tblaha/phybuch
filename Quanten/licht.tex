Man kann nicht genau sagen, was Licht, oder elektromagnetische Wellen allgemein, ist bzw. sind.

Es können aber Eigenschaften verschiedener Modelle, dem Wellenmodell und dem Teilchenmodell, auf Licht angewandt werden, sodass man sagen kann dass Licht etwas \glqq Drittes\grqq{} ist.

\begin{Wichtig}
	Wird manchmal angenommen: Licht ist keine Welle, bei der Photonen schwingen!
\end{Wichtig}

\subsection{Licht als Welle}

Bei der Wellenoptik wird Licht als Welle angenommen. Das Huygen'schen Prinzip ist zutreffend und Phänomene wie Brechung, Reflexion und Polarisation treten auf. Zudem gibt es Interferenzen an Spalten.

Eine Ausnahme bei der Optik ist allerdings die Tatsache, dass kurzwelligeres Licht stärker gebrochen wird, als Langwelliges. Dies ist auch nur als Quantenphänomen zu deuten.


\subsection{Licht als Teilchen}

Immer wenn von Licht als Photonen die Rede ist, betrachtet man die Teilcheneigenschaften des Lichts. Phänomene wären der Fotoeffekt; weitere, wie die Compton-Streuung oder die Existenz des Photonenimpulses, folgen in den nächsten Sektionen.