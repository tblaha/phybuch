Die Wellengleichung gibt die Elongation des Teilchens mit dem Abstand $x$ zum Ausgangspunkt beim Zeitpunkt $t$ an:

\begin{align} \label{eq:wellengleichung_y}
	y(x,t) = y_{max} \cdot sin{ (2\pi(\frac{t}{T}-\frac{x}{\lambda})) }
\end{align}

Zur Herleitung aus der Schwingungsgleichung muss Folgendes beachtet werden. Die Zeit $t_x$ bis das räumliche Teilchen $x$ von der Front der Welle erfasst wird, lässt sich recht einfach berechnen. Gebraucht wird dabei die Ausbreitungsgeschwindigkeit $c=\frac{\lambda}{T}$ und die generelle Formulierung der Geschwindigkeit $v=\frac{s}{t}$:

\begin{align}
\begin{split}
	t   &= \frac{s}{v} \\
	t_x &= \frac{x}{c} \\
	t_x &= \frac{x \cdot T}{\lambda}
\end{split}
\end{align}

Analog zur Verschiebung nach rechts von beispielsweise einer Parabel, muss der Term für $t_x$ von $t$ in der Schwingungsgleichung abgezogen werden. Damit ist die Variable $x$ mit in der Gleichung und der Term \glqq kompensiert\grqq{} den \glqq Offset\grqq{}, der sich durch das spätere Erfassen ergibt:

\begin{align}
\begin{split}
	y(x,t) &= y_{max} \cdot sin{[\omega \cdot (t-\frac{x}{\lambda} \cdot T)]} \\
	y(x,t) &= y_{max} \cdot sin{[\frac{2\pi}{T} \cdot (t-\frac{x}{\lambda} \cdot T)]} \\
	y(x,t) &= y_{max} \cdot sin{[2\pi \cdot (\frac{t}{T}-\frac{x}{\lambda \cdot T} \cdot T)]} \\
	y(x,t) &= y_{max} \cdot sin{[2\pi \cdot (\frac{t}{T}-\frac{x}{\lambda})]}
\end{split}
\end{align}