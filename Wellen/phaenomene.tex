Im Folgenden werden bei der Nennung von \glqq Wellen\grqq{} immer Transversalwellen gemeint, wenn es keine weitere Anmerkung gibt.  


\subsection[Ausbreitung in Elementarwellen]{Ausbreitung in Elementarwellen (Huygens'sches Prinzip)} \label{subsec:ausbreitung}

Das Huygens'sche Prinzip macht eine wichtige Aussage zur Ausbreitung von Wellen:

\glqq Jedes Teilchen (Oszillator), das von einer Wellenfront erfasst wird, löst von sich aus eine zirkulare Welle nach allen Seiten aus.\grqq{} Die eigentlich sichtbare Wellenfront ist nach Huygens eine Einhüllende aller \glqq Elementarwellen\grqq . Dadurch ist es Wellen unter anderem möglich in den geometrischen Schattenraum zu propagieren. \referenz{subsec:doppelspalt}


\subsection{Reflexion}  
%Cue for pics
	\subsubsection{Fixiertes Ende}
	
	Wenn eine Welle von einem fixierten Ende (eingespannt) reflektiert wird, dann gibt es einen Phasensprung von $180 \degree$ oder $\pi$, bzw $\frac{T}{2}$.
	
	\subsubsection{Loses Ende}
	
	Trifft eine Welle auf ein loses Ende, dann wird sie ohne Phasensprung reflektiert.



\subsection{Überlagerung} \label{subsec:ueberlagerung}

Wenn sich Wellen in einem räumlichen Punkt treffen, dann überlagern sie sich und bilden eine summierte Welle. Wenn also ein Wellenberg auf einen Wellenberg trifft, addieren sich beide Welle und der resultierende Wellenberg ist höher. Sollte auf einen Wellenberg ein Wellental treffen wird auf addiert, und die resultierende Welle von der Amplitude her kleiner.

Nachdem sich die Wellen in diesem Punkt überlagert haben, laufen beide weiter ohne die andere zu beeinflussen; so als hätte es die Überlagerung nie gegeben.

	\subsubsection{Stehende Welle}
	
	Wenn sich gegenläufige, das heißt parallel und identisch, aber in die entgegengesetzte Richtung fortschreitende, Wellen mit derselben Wellenlänge überlagern, entsteht eine stehende Welle. 
	Das heißt, dass es mindestens 2 Oszillatoren auf dieser Welle gibt, die sich gar nicht bewegen (\glqq stehen\grqq), die sogenannten Knotenpunkte.
	
	Eine stehende Welle kann zum Beispiel ausgelöst werden, wenn bei einer Reflexion am festen Ende die Hälfte der Wellenlänge $\lambda$ ein ganzzahliges Vielfaches der Abstands $l$ der beiden Enden ist. $k$ ist in diesem Fall eine Variable, die nur positive ganze Zahlen annehmen kann:
	
	\begin{equation} \label{stehendewelle}
		\lambda = 2k \cdot l \ \ \ wobei \ \ k \in 1,2,3...
	\end{equation}



\subsection{Interferenzen} \label{sec:interferenz}
%cue for pic
Wenn sich Wellen, die nicht nur dieselbe Wellenlänge, sondern auch die selbe Amplitude aufweisen, in einem Punkt überlagern, dann kann es zu Interferenzen kommen.

	\subsubsection{Konstruktive Interferenz}
	
	Wenn in dem Punkt beide Wellen ein Maximum aufweisen, sich also zu einem höheren Wellenberg addieren, spricht man von konstruktiver Interferenz.
	
	Dazu muss der Gangunterschied $\delta$ ein ganzzahliges Vielfaches der Wellenlänge sein:
	
	\begin{equation}	\label{eq:kon_interferenz}
		\delta = k \cdot \lambda \ \ \ wobei \ \ k \in 1,2,3...
	\end{equation}
	
	\subsubsection{Destruktive Interferenz}
	
	Im Gegensatz zur konstruktiven Interferenz, treffen nun also Wellen aufeinander, die einmal einen Wellenberg und einmal ein Wellental aufweisen. Da sie dieselbe Amplitude haben, ist die resultierende Amplitude im betroffenen Punkt $0$.
	
	Der Gangunterschied $\delta$ muss daher die Summe eines ganzzahliges Vielfaches Wellenlänge und der halben Wellenlänge sein:
	
	\begin{equation}
		\delta = k \cdot \lambda + \frac{\lambda}{2} \ \ \ wobei \ \ k \in 1,2,3...
	\end{equation}

	Zur einfacheren Umstellung nach $\lambda$ existiert diese, internationale Form:
	
	\begin{equation} \label{eq:des_interferenz}
		\delta = (2k -1) \cdot \frac{\lambda}{2} \ \ \ wobei \ \ k \in 1,2,3...
	\end{equation}