Als Welle bezeichnet man einen Vorgang bei dem Energie aber keine Masse transportiert wird. Es handelt sich um ein System gekoppelter Oszillatoren bei dem die Energie, die bei der Anregung eines Oszillators aufgewendet wird, vom diesem an die anderen Oszillatoren abgegeben wird. Daher wird die Energie räumlich transportiert, obwohl, absolut gesehen, sich jeder Oszillator noch stets an seinem Platz befindet.

\subsection{Transversalwelle}

Eine Transversalwelle ist eine Welle, deren Auslenkungsvektoren (= Richtung der Auslenkung) senkrecht auf der Ausbreitungsrichtung stehen.

Bsp.: Seilwelle, elektromagnetische Wellen (bei $ 400nm \leq \lambda \leq 700nm $: Licht)

\subsection{Longitudinalwelle}

Eine Longitudinal ist eine Welle, deren Auslenkungsvektoren (= Richtung der Auslenkung) parallel auf der Ausbreitungsrichtung stehen.

Bsp.: Schallwellen