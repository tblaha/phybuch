Bei einer Welle reicht \emph{ein} Diagramm zur Darstellung nicht aus. Da es mehrere Oszillatoren gibt, die alle zu unterschiedlichen Zeiten zu schwingen beginnen, bräuchte man ein Elongation-Zeit-Diagramm (\referenz{sec:diagramm_schwingung}) für jedes Teilchen, oder ein 3-dimensionales Diagramm, um die Schwingung aller Teilchen in Abhängigkeit \emph{eines} Zeitpunktes darzustellen.

Allerdings reichen zwei Diagramme um alle Kenngrößen der Welle erfassen und ablesen zu können: Ein Elongation-Zeit-Diagramm eines beliebigen Teilchens, welches die Schwingung dieses Teilchens über der Zeit darstellt und ein Elongation-Strecke-Diagramm (ein \glqq Standbild\grqq) zu einem beliebigen Zeitpunkt, welches die Auslenkung der Oszillatoren über der x-Position dieser Oszillatoren darstellt.

%Cue for example

Dazu muss angemerkt werden, dass hierbei nur eine Dimension betrachtet wird, eigentlich müsste die komplette räumliche Ausbreitung bei einem Elongation-Strecke-Diagramm berücksichtigt werden, in alle 3 Dimensionen. Zur Vereinfachung, damit man statt einem 4-Dimensionalen Diagramm ein 2-Dimensionales (x,y) Diagramm benutzen, wird nur \emph{eine} Ausbreitungsrichtung berücksichtigt.
