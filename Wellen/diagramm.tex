Einer Welle reicht ein Diagramm zur Darstellung nicht aus. Da es sehr viele Oszillatoren gibt, die alle zu unterschiedlichen Zeiten zu schwingen beginnen, bräuchte man ein Elongation-Zeit-Diagramm (\referenz{subsec:diagramm_schwingung}) für jedes Teilchen, oder ein 3-dimensionales Diagramm.

Allerdings reichen zwei Diagramme, davon ein Elongation-Zeit-Diagramm eines beliebigen Teilchens und ein Elongation-Strecke-Diagramm (auch bekannt als \glqq Standbild\grqq) zu einem beliebigen Zeitpunkt, um alle Kenngrößen der Welle erfassen zu können.