\subsection[Amplitude]{Amplitude: $y_{max}$ o. $s_{max}$ o. $\hat{y}$ o. $\hat{s}$ (Basiseinheit: $m$)}

\textbf{Kategorie: Schwingung \textit{eines} Oszillators}

\noindent Die maximale Elongation (=\glqq Auslenkung\grqq) der Schwingungen der einzelnen Oszillatoren.



\subsection[Periodendauer]{Periodendauer: $T$ (Basiseinheit: $s$)}

\textbf{Kategorie: Schwingung \textit{eines} Oszillators}
	
\noindent Die Zeit, die es dauert, bis \emph{einer} der schwindenden Körper an der selben Stelle von der selben Richtung aus angelangt ist. Beispielsweise vom positiven Schwingungsmaximum (\glqq Berg\grqq) zum nächsten oder von der Nullstelle (\glqq Ruhelage\grqq) zur 2. darauffolgenden Nullstelle.

Davon abgeleitet:
\begin{itemize}
	\item Frequenz: $f=\frac{1}{T}$ (Basiseinheit: $Hz=\frac{1}{s}$)

	Anzahl der Perioden pro Sekunde.
	\item Winkelgeschwindigkeit: $\omega=2 \pi f=\frac{2 \pi}{T}$ (Basiseinheit: $\frac{rad}{s}$)

	Änderung des Winkels über der Zeit, wobei eine ganze Periode mit $360 \degree$ im Grad oder, im Physikunterricht verwendet, mit $2 \pi$ im Bogenmaß (eng: \glqq Radian\grqq) bezeichnet wird.\footnote{Umrechnung des Winkels $\alpha$ von Grad nach Bogenmaß: $\alpha_{rad} = \alpha_{deg} \cdot \frac{2\pi}{360 \degree} = \alpha_{deg} \cdot \frac{\pi}{180 \degree} $}
\end{itemize}



\subsection[Wellenlänge]{Wellenlänge: $\lambda$ (Basiseinheit: $m$)}

\textbf{Kategorie: Oszillatorsystem}

\noindent Der räumliche Abstand zwischen 2 Wellenbergen im Elongation-Strecke-Diagramm, der Abstand eines Nulldurchlaufs mit dem zweiten darauffolgenden.



\subsection[Ausbreitungsgeschwindigkeit]{Ausbreitungsgeschwindigkeit: $c$ (Basiseinheit: $\frac{m}{s}$)}

\noindent Die Ausbreitungsgeschwindigkeit ist eigentlich keine echte Kenngröße, da sie nicht in der Wellengleichung auftaucht. Trotzdem ist sie hilfreich um $\lambda$ oder $f$ zu berechnen. Da eine Geschwindigkeit generell als Strecke pro Zeit ($v=\frac{s}{t}$) definiert ist, ist die Wellenlänge die Strecke, die die Wellenfront nach einer Periode ($T$) zurückgelegt hat.

\begin{equation} \label{eq:wellen_c}
	c=\frac{\lambda}{T}=\lambda \cdot f
\end{equation}