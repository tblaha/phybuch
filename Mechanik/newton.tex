Axiome sind grundlegende Annahmen, die nicht bewiesen wurden, eventuell sogar nicht bewiesen werden können, aber trotzdem die Grundlage einer These bilden. Man nimmt also an, dass diese wahr sind.

Die bekanntesten Axiome lieferte Isaac Newton um 1687, welche wiederum die Grundlage für die klassische Mechanik bilden. Bis heute gab es keinen Einspruch betreffend ihrer Gültigkeit.

\subsection{1. Axiom -- Trägheitsprinzip}

\begin{quote}
\glqq Ein Körper verharrt im Zustand der Ruhe oder der gleichförmigen Bewegung (Siehe: \ref{subsec:gleichfoermig}), sofern er nicht durch einwirkende Kräfte zur Änderung seines Zustands gezwungen wird.\grqq \footnote{Von Wikipedia, die freie Enzyklopädie: \url{https://de.wikipedia.org/wiki/Newtonsche_Gesetze}}
\end{quote}

\noindent Es wird deutlich, dass ein Körper nicht von sich selbst aus seine Bewegungsrichtung oder Geschwindigkeit ändern kann, sondern dazu eine äußere Kraft notwendig ist.

\begin{Beispiel}
	Auf eine Rakete, die im Weltraum in der völligen Luftleere und völliger Schwerelosigkeit schwebt, wirkt keine Kraft ein, welcher Natur auch immer, sodass sie sich immerfort mit der selben Geschwindigkeit in die selbe Richtung bewegt. Gesteuert würde sie über das \glqq Abwerfen\grqq{} von Masse, beispielsweise durch Gase, die unter Druck stehen und dann aus einer bestimmten Öffnung in aus der Rakete strömen.
\end{Beispiel}


\subsection{2. Axiom -- Aktionsprinzip}

\begin{quote}
\glqq Die Änderung der Bewegung ist der Einwirkung der bewegenden Kraft proportional und geschieht nach der Richtung derjenigen geraden Linie, nach welcher jene Kraft wirkt.\grqq \footnote{Von Wikipedia, die freie Enzyklopädie: \url{https://de.wikipedia.org/wiki/Newtonsche_Gesetze}}
\end{quote}

\noindent Eine Änderung einer Bewegung, also die Änderung der Geschwindigkeit, ist in physikalischen Termen auch die \glqq Beschleunigung\grqq{} mit Formelzeichen $a$. Daher gilt:

\begin{align}
	F \sim a
\end{align}

\noindent Hierbei steht das Zeichen $\sim$ für \glqq proportional zu\grqq . 

Leonard Euler formulierte erst 60 Jahre später die Gleichung, die man heute mit dem 2. Axiom verbindet und gleichzeitig ein extrem wichtiger Grundsatz für unzählige physikalische Gesetzmäßigkeiten ist: \glqq Kraft ist Masse mal Beschleunigung\grqq :

\begin{align}	\label{eq:2.axiom}
	F = m \cdot a
\end{align}


\subsection{3. Axiom -- Actio und Reactio}

\begin{quote}
\glqq Kräfte treten immer paarweise auf. Übt ein Körper A auf einen anderen Körper B eine Kraft aus (actio), so wirkt eine gleich große, aber entgegen gerichtete Kraft von Körper B auf Körper A (reactio).\grqq \footnote{Von Wikipedia, die freie Enzyklopädie: \url{https://de.wikipedia.org/wiki/Actio_und_Reactio}}
\end{quote}

\noindent Oder, mathematisch ausgedrückt: Wenn Körper $A$ eine Kraft auf Körper $B$ auswirkt, dann gilt:

\begin{align}
	F_{A \ auf \ B} = -F_{B \ auf \ A}
\end{align}

\begin{Beispiel}
	Wenn die Rakete aus dem Beispiel des ersten Axioms gesteuert werden soll, wird Masse abgeworfen. Dieses Abwerfen erfordert eine Kraft, die sich aus dem 2. Axiom ergibt (die Masse wird ja beschleunigt, wenn sie abgeworfen wird. Daraus ergibt sich dann, gemäß $F=m \cdot a$ eine Kraft). Auf die Rakete wirkt dann dieselbe Kraft, allerdings in die andere Richtung, sodass sie durch diese Kraft beschleunigt wird und ihre Bewegungsrichtung ändert.
\end{Beispiel}





