Die Gesetze der Kreisbewegung beschreiben einen Körper, der sich mit einer konstanten Geschwindigkeit auf einer Kreisbahn mit konstantem Radius bewegt.


\subsection{Frequenz}

Die Frequenz $f$ bezeichnet die Anzahl an Kreisumläufen pro Zeiteinheit. Die Einheit ist $Hz$ (sprich: \glqq Hertz\grqq ), also $\frac{1}{s}$.

Die Dauer \emph{eines} Kreisumlaufes nennt man auch die Periodendauer $T$. Dann gilt:

\begin{align}
	f = \frac{1}{T}
\end{align}

\subsection{Winkelgeschwindigkeit}

Die Winkelgeschwindigkeit gibt an, wie groß der Umfangsabschnitt ist, der in einer bestimmten Zeit absolviert wird:

\begin{align}
	\omega = 2 \pi f = \frac{2 \pi}{T}
\end{align}

\noindent Die Einheit ist $\frac{rad}{s}$ (sprich: \glqq Radiant pro Sekunde\grqq ) und ein Umlauf entspricht $2\pi$.

\begin{Wichtig}
Das Formelzeichen $\omega$ ist \glqq klein Omega\grqq{} und nicht \glqq w\grqq .
\end{Wichtig}


\subsection{Bahngeschwindigkeit im Kreis}

Die Bahngeschwindigkeit gibt in absoluten Einheiten ($\frac{m}{s}$) an, wie schnell sich das Objekt auf der Bahn fortbewegt. Zusätzlich zur Winkelgeschwindigkeit, muss bei der Kreisbahn daher noch der Radius bekannt sein:

\begin{align} \label{eq:bahngeschwindigkeit}
	v=\omega r= 2r\pi f = \frac{2r\pi}{T}
\end{align}

Eine andere Herleitung aus dem Kreisumfang $U=2r\pi$, der Periodendauer $T$ und der generellen Formel für Bahngeschwindigkeit $v=\frac{s}{t}$ könnte wie folgt Aussehen:

\begin{align}
	v &=\frac{s}{t} \\
	v &=\frac{2r\pi}{T}
\end{align}


\subsection{Zentripetalbeschleunigung}
%Cue for herleitung
Um auf einer Kreisbahn zu bleiben, muss eine Zentripetalbeschleunigung auf einen Körper wirken, die zum Kreiszentrum hin zeigt. Die Formel lautet:

\begin{align}
	a_z=\frac{v^2}{r}=\omega^2 r
\end{align}

Eine tolle Herleitung findet sich \href{http://www.schule-bw.de/unterricht/faecher/physik/online\_material/mechanik2/kreis/zentripetalkraft.htm}{hier}\footnote{\url{http://www.schule-bw.de/unterricht/faecher/physik/online_material/mechanik2/kreis/zentripetalkraft.htm}}.

\subsection{Zentripetalkraft}

Dies ist die Kraft, die auf einen Körper ausgewirkt werden muss, damit er auf einer Kreisbahn bleibt. Zusätzlich zur Zentripetalbeschleunigung muss nun also noch gemäß Newtons zweiten Axiom $F=a \cdot m$ die Masse als Faktor in die Gleichung aufgenommen werden:

\begin{align}
	F_z &= a_z \cdot m \\
	F_z &= \frac{v^{2}m}{r}
\end{align}
