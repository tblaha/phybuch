Als Stöße bezeichnet man das Aufeinandertreffen von zwei Körpern unterschiedlicher Geschwindigkeiten. Das Ziel ist es, aus deren Massen und Geschwindigkeiten und der Art des Stoßes auf die Geschwindigkeiten \emph{nach} dem Stoß zu schließen.

\subsection{Impuls} \label{subsec:impuls}

Zur Beschreibung von Stößen muss vorerst noch eine weitere mechanische Größe eingeführt werden; der Impuls $p$. Der Impuls ist das Produkt aus Masse und Geschwindigkeit eines Körpers:

\begin{align}
	\vec{p} = m \cdot \vec{v}
\end{align}

\noindent Damit ist die Einheit des Impulses $\frac{kg \cdot m}{s} = Ns$ (sprich: \glqq Newtonsekunde\grqq).

\begin{NiceToKnow}
Der Impuls ist das zeitliche Integral der Kraft: $\int F \ dt = \int m \cdot a \ dt = m \cdot v$
\end{NiceToKnow}


\subsection{Elastischer Stoß}

Aus den zwei Stoßarten (elastisch und unelastisch) ist der elastische Stoß der ideale Stoß, bei dem keine Reibung oder Verformung zwischen den beiden aufeinandertreffenden Körpern stattfindet. In der Praxis geschieht das natürlich trotzdem, aber bspw. sind Stahlkugeln in guter Näherung reibungsfrei, sodass bei diesen und ähnlichen Versuchen von einem elastischen Stoß ausgegangen werden kann.

Da keine Energie durch Reibung in Wärme oder Verformung umgewandelt wird und somit \glqq verloren geht\grqq , muss neben dem Impuls auch die Energie erhalten bleiben, also während des Versuches konstant sein. Daher gestaltet sich die Herleitung für eine Gleichung, die die Geschwindigkeiten nach dem Stoß voraussagt, recht komplex; beide Sätze (Energie- und Impulssatz) müssen berücksichtigt werden. Daher wird hier das Gesetz nur genannt:

Wenn ein Körper mit der Masse $m_1$ und der Geschwindigkeit $v_1$ auf einen zweiten Körper mit der Masse $m_2$ und der Geschwindigkeit $v_2$ unter den Voraussetzungen eines elastischen Stoßes trifft, gilt für die Geschwindigkeit des zweiten Körpers nach dem Stoß $v_2'$:

\begin{align}
\begin{split}
	v_2'=\frac{m_2 \cdot v_2 + m_1(2v_1 - v_2)}{m_1 + m_2}
\end{split}
\end{align}

\noindent Für $v_1'$, also Körper 1, gilt dann:

\begin{align}
\begin{split}
	v_1'=\frac{m_1 \cdot v_1 + m_2(2v_2 - v_1)}{m_1 + m_2}
\end{split}
\end{align}


\subsection{Unelastischer Stoß}

Bei einem unelastischen Stoß geht ein Teil der kinetischen Energie der zwei Stoßkörper in Wärme oder Verformung \glqq verloren\grqq , dass heißt sie steckt nachher nicht mehr in der kinetischen Energie der Körper. 

Bei einem perfekt unelastischen Stoß, bei dem die beiden Körper sich \glqq verkeilt\grqq{} zusammen mit der selben Geschwindigkeit fortbewegen, muss der Impulserhaltungssatz trotzdem erhalten bleiben. Aus diesem geht für $v'$ nach dem Stoß hervor:

\begin{align}
\begin{split}
	p' &= p_1 + p_2 \\
	v'(m_1 + m_2) &= v_1 \cdot m_1 + v_2 \cdot m_2 \\
	v' &= \frac{v_1 \cdot m_1 + v_2 \cdot m_2}{m_1 + m_2}
\end{split}
\end{align}

