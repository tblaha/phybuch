Als Stöße bezeichnet man das Aufeinandertreffen von zwei Körpern unterschiedlicher Geschwindigkeiten. Das Ziel ist es, aus deren Massen und Geschwindigkeiten und der Art des Stoßes auf die Geschwindigkeiten \emph{nach} dem Stoß zu schließen.

\subsection{Impuls} \label{subsec:impuls}

Zur Beschreibung von Stößen muss vorerst noch eine weitere mechanische Größe eingeführt werden, der Impuls $p$. Der Impuls ist das Produkt aus Masse und Geschwindigkeit eines Körpers:

\begin{align}
	\vec{p} = m \cdot \vec{v}
\end{align}

\noindent Damit ist die Einheit des Impulses $\frac{kg \cdot m}{s} = Ns$ (sprich: \glqq Newtonsekunde\grqq).

\begin{NiceToKnow}
Der Impuls ist das zeitliche Integral der Kraft: $\int F \ dt = \int m \cdot a \ dt = m \cdot v$
\end{NiceToKnow}


\subsection{Elastischer Stoß}

Aus den zwei Stoßarten (elastisch und unelastisch) ist der elastische Stoß der ideale Stoß, bei dem keine Reibung zwischen den beiden aufeinandertreffenden Körpern stattfindet. In der Praxis geschieht das natürlich trotzdem, aber bspw. sind Stahlkugeln in guter Näherung reibungsfrei, sodass bei diesen und ähnlichen Versuchen von einem elastischen Stoß ausgegangen werden kann.

Da keine Energie durch Reibung in Wärme oder Verformung umgewandelt wird und somit \glqq verloren geht\grqq , muss neben dem Impuls auch die Energie erhalten bleiben, also während des Versuches konstant sein. Wenn die Indices $_1$ bzw. $_2$ für die Körper $1$ bzw. $2$ stehen, müssen bei einem elastischen Stoß folgende Gesetze gelten:

\begin{align}
\begin{split}
	p_{ges} &= p_{ges}' \\
	p_1 + p_2 &= p_1' + p_2'
\end{split}
\end{align}

\noindent und:

\begin{align}
\begin{split}
	E_{ges} &= E_{ges}' \\
	\frac{1}{2} m_1 \cdot v_1^2 + \frac{1}{2} m_2 \cdot v_2^2 &= \frac{1}{2} m_1' \cdot v_1'^2 + \frac{1}{2} m_2' \cdot v_2'^2
\end{split}
\end{align}

