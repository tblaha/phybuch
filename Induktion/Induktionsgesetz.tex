In einer Spule: Wenn $N$ die Anzahl der Windungen einer Spule ist und $\phi$ das Produkt aus der Flussdichte $B$ des Magnetfeldes und der Fläche $A$ des Spulenquerschnitts, welche vom Magnetfeld durchsetzt ist (also $\phi = B \cdot A$), gilt für die Induktionsspannung $U_{Ind}$:

\begin{align} \label{eq:InduGe}
	U_{ind} = -N \frac{d\phi}{dt}
\end{align}

\noindent Für die Induktionsspannung gilt für bewegte, gerade, nicht geschlossene Leiter der Länge $l$ mit $\vec{l} \perp \vec{B} \perp \vec{v}$ (Längsachse des Leiters senkrecht zu den Feldlinien und zur Bewegungsrichtung; Bewegungsrichtung senkrecht zu den Feldlinien)(Siehe: \referenz{subsec:handregeln}):

\begin{align}
	U_{ind} = -l \cdot B \cdot v
\end{align}