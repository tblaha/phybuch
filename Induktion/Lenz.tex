Die Lenz'sche Regel besagt, dass eine Spannung immer dem induzierenden Magnetfeld entgegengesetzt induziert wird. Daher stammen auch die Negationen in den Induktionsgesetzen.

Da eine Spule, die sich in einem geschlossenen Stromkreis befindet, und in der eine Spannung induziert wurde, ein Strom fließt, der selbst ein Magnetfeld erzeugt, muss dieses Magnetfeld, gemäß der Energieerhaltung dem \emph{induzierenden} Magnetfeld entgegengesetzt sein. Beispielsweise würde sonst ein Magnet, der sich durch eine Spule bewegt, durch das Magnetfeld der Spule, das er selbst induzierte, noch immer weiter beschleunigt werden; ein unmögliches \emph{Perpetuum mobile}.