Die Lenz'sche Regel besagt, dass eine Spannung immer dem Magnetfeld entgegengesetzt induziert wird. Daher stammen auch die Negationen in den Induktionsgesetzen.

Da eine Spule in einem geschlossenen Stromkreis, in der eine Spannung induziert wurde, ein Strom fließt, der selbst ein Magnetfeld erzeugt, muss dieses Magnetfeld, gemäß der Energieerhaltung dem induzierenden Magnetfeld entgegengesetzt sein. Beispielsweise würde sonst ein Magnet, der sich durch eine Spule bewegt, durch das Magnetfeld der Spule, das er selbst induzierte, noch immer weiter beschleunigt werden; ein unmögliches perpetuum Mobile.