Die Induktivität beschreibt die Größe des Vermögens von elektrischen Leitern, insbesondere Spulen, zu Induzieren.

\subsection{Definition und Gesetz}

Für lange Spulen gilt mit der Querschnittsfläche $A$:

\begin{equation} \label{eq:Invitat}
	L = \mu_0 \mu_r \frac{N^2 \cdot A}{l}
\end{equation} 


\subsection{Induktionsgesetz mit der Induktivität}

Nach Einsetzen der magnetischen Flussdichte (siehe Gleichung \ref{eq:MaFluss} auf Seite \pageref{eq:MaFluss}) in die Formel für Induktionsspannung (siehe Gleichung \ref{eq:InduGe} auf Seite \pageref{eq:InduGe}) ergibt sich:

\begin{equation}
	U_{ind} = - \mu_0 \mu_r \cdot \frac{N^2}{l}
				\cdot A \cdot \frac{dI}{dt}
\end{equation}

Auffallend ist, dass der Teil $\mu_0 \mu_r \cdot \frac{N^2}{l} \cdot A$ die eben benannte Induktivität ist (siehe Gleichung \ref{eq:Invitat}). Daher kann die Induktionsspannung auch wie folgt berechnet werden:

\begin{equation}
	U_{ind} = - L \cdot \frac{dI}{dt}
\end{equation}