\subsection{Definition}

Lange Spulen sind gerade und deren Länge ist größer als deren Durchmesser. Damit ist im Inneren ein homogenes Magnetfeld zu beobachten, siehe \referenz{sec:feldlinienbilder}.


\subsection{Gesetze}

Die magnetische Flussdichte im Inneren einer stromdurchflossenen Spule in definiert als:

\begin{equation} \label{eq:MaFluss}
	B = 	\mu_0 \mu_r \cdot N \cdot \frac{I}{l}
\end{equation}

\noindent Hierbei ist $\mu_0$ die magnetische Permeabilitätskonstante und $mu_r$ die relative Permeabilitätszahl, welche für Luft und Vakuum $\approx 1$ ist. 

\begin{NiceToKnow}
Mit speziellen Legierungen können relative Permeabilitäten von über $9 \cdot 10^5$ erreicht werden, was die Flussdichte um den selben Faktor erhöhen würde.
\end{NiceToKnow}