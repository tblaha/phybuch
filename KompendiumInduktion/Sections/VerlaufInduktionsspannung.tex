Sollte sich bei einem Experiment ein Permanentmagnet durch eine Spule, die länger als er selbst ist, bewegen, steigt die Induktionsspannung beim Eintritt schnell an. Sobald sich mehr als die Hälfte der Fläche des Magneten in der Spule befindet, nimmt die Spannung ab und solange er vollständig in der Spule ist beträgt die Spannung 0, da sich das Magnetfeld in der Spule nun nicht mehr ändert. Beim Austreten nimmt die Spannung wieder zu, diesmal jedoch in der entgegengesetzten Richtung.