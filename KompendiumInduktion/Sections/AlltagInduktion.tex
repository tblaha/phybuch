%%%%% Physik Kompendium Nr.2 %%%%%
%% 03 -- Alltägliche Anwendung der Induktion %%


\subsection{Wirbelstrombremse}

\subsubsection{Funktionsprinzip}
% Cue for review Strom vs Spannung
In einer Wirbelstrombremse induziert ein magnetisches Feld einen Strom in einen, sich durch dieses Feld bewegenden, ferromagnetischen Körper, der wiederum in diesem Körper einen Wirbelstrom bildet. Dieser baut ein Magnetfeld auf, welches, gemäß Lenz'scher Regel, in die entgegengesetzt Richtung zeigt. Damit ziehen sich das sogenannte Bremsschwert und der induzierende Magnet an.

Die Wirkung der Bremse ist proportional abhängig von der Fläche, Induktivität und Geschwindigkeit des Schwertes sowie von der Flussdichte des Magnetfeldes.


\paragraph{Merkmale}

\begin{itemize}

	\item Geschwindigkeitsabhängig
	
Da in der Formel für die Induktionsspannung ($U_{ind}=-L \cdot \frac{dI}{dt}$) die Zeit für die Änderung der Flussdichte im Nenner steht, ist die Bremswirkung bei höheren Geschwindigkeiten höher, Gesetzt dem Fall, dass sich die Fläche des Bremsschwertes und die Stromstärke nicht ändert.

\end{itemize}


\paragraph{Vorteile}

\begin{itemize}
	\item Verschleiß- und Wartungsfreiheit
	
Da bei einer Wirbelstrombremse keine Teile aufeinander Schleifen, ist diese weitgehend frei von Verschleiß und auch deutlich Wartungsfreundlicher.

Lediglich eine zu hohe Hitzeentwicklung sollte verhindert werden.

	\item Ausfallsicherheit
	
Während herkömmliche Bremsen abhängig von Hydraulik oder Kabelzügen etc. sind, funktionieren Wirbelstrombremsen allein mit Strom, oder, wenn sie mit Permanentmagneten statt Elektromagneten ausgestattet sind, komplett Ausfallsicher.

\end{itemize}

\paragraph{Nachteile}

\begin{itemize}

	\item Kein vollständiges Abbremsen möglich
	
Aufgrund der Abhängigkeit von der Geschwindigkeit, ist es theoretisch nicht möglich, allein mit einer Wirbelstrombremse einen Gegenstand zum Stillstand zu bringen.

\end{itemize}


\subsubsection{Typische Anwendungen}

Die Wirbelstrombremse findet als klassische Bremse für Fahrzeuge häufig Anwendung in Achterbahnen und Trams, da diese oft mit einer vorhersehbaren Geschwindigkeit unterwegs sind, und daher die Bremswirkung genau berechnet werden kann. Dies ist für die Wirbelstrombremse recht wichtig, da, wie schon erwähnt, die Bremswirkung von der Geschwindigkeit abhängig ist.

In Freefalltowern werden Wirbelstrombremsen mit Permanentmagneten eingesetzt, um Sicherheit auch bei einem plötzlichen Stromausfall zu gewährleisten.

In LKWs unterstützten sie manchmal die herkömmliche Bremse.

In Hometrainern hält eine induktionsbasierte Bremse das Gerät Wartungsfreier und leiser.


\subsection{Induktionsherd}

\subsubsection{Funktionsprinzip}

Unter der Herdplatte sitzt eine Spule, durch die ein hochfrequenter Wechselstrom fließt. Dadurch entsteht ein sich ständig änderndes Magnetfeld, welches dann ständig einen Strom in den Boden des Topfes induziert und in diesem für Wärmeentwicklung sorgt.

Die Stärke ist in der Regel proportional Abhängig von der gewählten Frequenz und Stärke des Magnetfeldes, sowie des Materials des Topfes.


\paragraph{Vorteile}

\begin{itemize}
	\item Bessere Regelbarkeit und Ansprache
	
Die Wärmeentwicklung lässt sich sehr gut Steuern und der Herd spricht sehr schnell an, was ihn optimal für große Küchen macht, in denen dies erforderlich ist.

	\item Geringe Wärmeentwicklung auf der Herdplatte
	
Da die Hitze sich erst im Topf entwickelt, ist die Herdplatte deutlich weniger heiß als bei einem herkömmlichen Herd, was vor Verbrennungen schützt.

	\item Effizienz
	
Ein Induktionsherd ist deutlich effizienter als ein herkömmlicher Elektroherd, da keine Energie in die Aufheizung der Platte gesteckt werden muss.
\end{itemize}



\paragraph{Nachteile}

\begin{itemize}

	\item Neue Töpfe

Natürlich benötigen die Töpfe bzw. Pfannen einen ferromagnetischen Boden, was bei herkömmlichen Töpfen häufig nicht gegeben ist. Daher muss eventuell viel Geld für ein neues Topf-/Pfannenset ausgegeben werden.
\end{itemize}

