\subsection{Grundprinzip}

Wenn Elektronen durch einen Leiter fließen, erzeugen sie einen Strom, der in einer Spule ein Magnetfeld aufbaut. Beim Aufbau dieses Magnetfeldes ändert sich dieses logischerweise. Diese Änderung erzeugt allerdings in der selben Spule eine so genannte Selbstinduktion, deren Strom gemäß der Lenz'schen Regel (siehe Sektion \ref{sec:Lenz} auf Seite \pageref{sec:Lenz}) entgegengesetzt der eigentlichen Stromrichtung gerichtet ist.

Dies führt dazu, dass der Stromfluss "gebremst" wird.

\subsection{Verhalten beim Unterbrechen eines Stromkreises}

Wenn in einem Stromkreis eine Spule mit Strom durchflossen wurde und sich in ihr das Magnetfeld aufgebaut hat, wird bei einer abrupten Unterbrechung des Stromkreises das Magnetfeld zusammenbrechen und aufgrund dieser starken Änderung eine hohe, verpolte Spannung induziert.

Dies lässt sich mit einer parallel geschalteten Glimlampe veranschaulichen, die eine Auslösespannung von circa 60 - 70 Volt besitzt. Sie wird beim Ausschalten leuchten, auch wenn die Spannung bisher nur wenige Volt betrug.