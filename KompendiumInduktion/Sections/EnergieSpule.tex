%%%%% Physik Kompendium Nr.2 %%%%%
%% 01 -- Energie des Magnetfeldes einer Spule %%


Die Energie, die in Form eines Magnetfeldes in einer Spule gespeichert ist, ist wie folgt zu berechnen:

\begin{equation} \label{eq:EnergieSpule}
	E = \frac{1}{2} L \cdot I^2
\end{equation}

Eine mögliche Herleitung, die eventuell interessant, aber sicher nicht Klausurrelevant ist, ist diese. Sie beruht auf der allgemeinen und der elektrischen Leistung ($P = \frac{\mathrm{d}E}{\mathrm{d}t}$ und $P = U \cdot I$) sowie auf der Induktivitätsbeziehung $U = L \cdot \frac{\mathrm{d}I}{\mathrm{d}t} <=> L = U \cdot \frac{\mathrm{d}t}{\mathrm{d}I}$:

\begin{align*}
\begin{split}
	P &= U \cdot I = \frac{\mathrm{d}E}{\mathrm{d}t} \\
	\mathrm{d}E &= U \cdot I \cdot \mathrm{d}t \\
	\frac{\mathrm{d}E}{\mathrm{d}I} &= U \cdot \frac{\mathrm{d}t}{\mathrm{d}I} \cdot I \\
	\frac{\mathrm{d}E}{\mathrm{d}I} &= L \cdot I \\
	E &= \int L \cdot I \ \mathrm{d}I \\
	E &= \frac{1}{2}L \cdot I^2
\end{split}
\end{align*}