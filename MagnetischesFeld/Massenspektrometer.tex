Beim Massenspektrometer macht man sich zur Identifikation von chemischen Elementen und deren Isotopen die unterschiedlichen Atommassen zu Nutze. Die ersten funktionstüchtigen Massenspektrometer gibt es seit dem frühen 20. Jahrhundert.

\subsection{Aufbau}

In der Ionenquelle werden Atome aus der Probe gelöst und ionisiert. Das heißt, sie sind nun im gasförmigen Zustand und zudem positiv geladen.

Nach einer Beschleunigungseinheit und einem Wien'schen Filter werden sie in ein homogenes Magnetfeld geleitet in welchem sie dann, gemäß der Lorentzkraft auf eine Kreisbahn gezwungen werden. 

\subsection{Mathematisierung}

Da man absolute Gewissheit über Ladung und Geschwindigkeit hat und auch die Flussdichte des Magnetfeldes kennt, ist die einzige Variable, die den Auftreffpunkt auf der Indikatorplatte bestimmt, die Masse des Atoms $m_a$. Wie beim Fadenstrahlrohr (Siehe \referenz{sec:Fadenstrahlrohr}) wird der Ansatz $F_{Zp} = F_{Lr}$ vollzogen:

\begin{align}
\begin{split}
	F_{Zp} 				  	&= F_{Lr} \\
	m_a \cdot \frac{v^2}{r} &= q \cdot B \cdot v \\
	m_a 					&= \frac{q \cdot B \cdot r}{v}
\end{split}
\end{align}










