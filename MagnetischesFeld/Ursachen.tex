\subsection{Ferromagnetisches Metall}

Ein magnetisches Feld wird im klassischsten Sinne ausgelöst, wenn ein ferromagnetisches Metall \glqq magnetisiert\grqq{} wird, was heißt, dass sich sogenannte Elementarmagnete ausrichten.\endnote{Siehe: \url{https://de.wikipedia.org/wiki/Elementarmagnet}}


\subsection{Elektromagnet}

Jegliche Leiter, auch nicht-ferromagnetischer Natur, lösen ein magnetisches Feld aus, wenn Strom durch sie fließt.

Wenn Strom (also Elektronen, die kleinste Einheit der negativen Ladung) durch einen Leiter fließt, ergibt sich ein magnetisches Feld, dessen Ausrichtung gemäß verschiedener Handregeln erfolgt. Siehe dazu \referenz{subsec:Faustregel}.


\subsection{Erdmagnetfeld}

Auch von der Erde geht ein magnetisches Feld aus, welches z.B. für Kompanten genutzt wird.

