\subsection{Feldlinien}

Wie schon beim elektrischen Feld wird hier ein Feldlinienmodell genutzt um Eigenschaften wie Ausrichtung (Polung) oder Stärke darzustellen. 

Magnetische Felder haben 2 Pole, die nicht negativ oder positiv, wie beim elektrischen Feld, sondern Nord- und Südpol genannt werden. Ungleichnamige Pole ziehen sich an, gleichnamige stoßen sich ab. Im Feldlinienmodell zeigen die Pfeile immer zum Südpol. Bei der Wechselwirkung zweier Körper gibt es eine abstoßende Kraftwirkung bei gegenläufigen Feldlinien und eine anziehende Kraftwirkung, wenn die Feldlinien in dieselbe Richtung zeigen.

Charakteristische Bilder von Feldlinien sowie Erklärungen zu diesen finden sich in \referenz{sec:feldlinienbilder}.


\subsection{Magnetische Flussdichte}

Auch beim magnetischen Feld gibt die Dichte der Feldlinien eine Größe für die Stärke des Feldes an. Diese Größe heißt aber nicht Feldstärke sondern \glqq magnetische Flussdichte \grqq{} (Formelzeichen $B$, Einheit \glqq Telsa\grqq : $T=\frac{kg}{As^2}=\frac{Vs}{m^2}$) (die magnetische Feldstärke gibt es auch, bezeichnet aber etwas anderes!). Diese ist also äquivalent zur Feldstärke des elektrischen Feldes und ordnet über die Lorentzkraft (Siehe: \referenz{sec:lorentzkraft}) jedem bewegten, geladenen Teilchen eine Kraft zu (Siehe \referenz{subsec:BLorentzDefinition}).


\subsection{Homogenes Feld} \label{subsec:MFeldHomogen}

Ein homogenes Magnetfeld hat die Eigenschaft, dass die Flussdichte in jedem Punkt gleich ist, was Berechnungen erheblich einfacher macht (Vergleiche: \referenz{subsec:EFeldHomogen}). Homogene Felder treten zum Beispiel im Inneren von Spulen oder zwischen den Schenkeln eines Hufeisenmagneten auf. Mehr dazu in \referenz{sec:feldlinienbilder}.

















